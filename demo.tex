% \documentclass[type = bachelor]{whu-thesis}
% \documentclass[type = master]{whu-thesis}
% \documentclass[type = master]{whu-thesis}
\documentclass[type = master,showframe]{whu-thesis}
% \documentclass[type = doctor,showframe]{whu-thesis}
% \documentclass[type = bachelor,showframe]{whu-thesis}
\whusetup{
  info = {
    title      = {论文题目},
    title*     = {A QoS-aware Self-Adaptive System Approach in Service Computing Environment},
    department = {数学与统计学院},
    department* = {School of Mathematics and Statistics},
    % major      = {数学与应用数学},
    author     = {张三},
    author*    = {Xxxxxx Xxxxx},
    student-id = {2021202012345},
    supervisor = {李某某},
    supervisor* = {Xxx Xxxxx},
    supervisor-outer = {王某某},
    academic-title = {教授},
    academic-title* = {Prof},
    academic-title-outer = {高级工程师},
    research-area = {英语语言学},
    subject = {英语},
    major   = {英国语言文学},
    research-area* = {Operator Theory},
    % doctor-cover-withsubject = false,
    % doctor-cover-info-spread = false,
    year = 2024,
    month = 1,
    cover-en-type = cs,
    % bachelor-english-cover=false
    clc = O175.29,
    udc = 517.9,
  },
  style = {
    font = times,
    math-font = termes,
    cjk-font = mac,
    cjk-fakefont = true,
  }
}
\usepackage{zhlipsum}
\usepackage{lipsum}
\begin{document}


\begin{abstract}
  \zhlipsum[1]
\end{abstract}

\begin{abstract*}
  \lipsum[1]
\end{abstract*}

\tableofcontents


\begin{notation}
  $\alpha$ & 常量 \\
  $\beta$ & 常量 \\
  $\gamma$ & 变量 \\
\end{notation}



% 正文
\mainmatter

\chapter{测试}

\section{文字测试}

这是武汉大学学位论文模版,欢迎使用。

This is Wuhan University thesis template, welcome to use!

\section{公式}

\subsection{算符、希腊字母}

\[\sum\prod\int\iint\alpha\beta\gamma\xi\zeta\eta\epsilon\varepsilon\theta\vartheta
  \phi\varphi\psi\]


\subsection{几类数学字母表}

\begin{itemize}
  \item \verb|\mathcal|: $\mathcal{ABCDEFGHIJKLMNOPQRSTUVWXYZ}$
  \item \verb|\mathscr|: $\mathscr{ABCDEFGHIJKLMNOPQRSTUVWXYZ}$
  \item \verb|\mathbb|: $\mathbb{ABCDEFGHIJKLMNOPQRSTUVWXYZ}$
\end{itemize}


\subsection{(不)带编号单行公式}

\begin{equation}
  a^2 + b^2 = c^2.
\end{equation}

\[ a^2 + b^2 = c^2.\]

\subsection{(不)带编号多行公式}

\begin{align}
  \text{sum} & = 1 + 2 + \cdots + n \\
             & = \frac12 n(n+1).
\end{align}

\begin{align*}
  \text{sum} & = 1 + 2 + \cdots + n \\
             & = \frac12 n(n+1).
\end{align*}

\subsection{矩阵}

\[\begin{pmatrix}
  a_{11} & a_{22} & a_{33} \\
  a_{21} & a_{22} & a_{23} \\
  a_{31} & a_{32} & a_{33} \\
\end{pmatrix} \quad
\begin{vmatrix}
  a_{11} & a_{22} & a_{33} \\
  a_{21} & a_{22} & a_{23} \\
  a_{31} & a_{32} & a_{33} \\
\end{vmatrix} \quad
\begin{bmatrix}
  a_{11} & a_{22} & a_{33} \\
  a_{21} & a_{22} & a_{23} \\
  a_{31} & a_{32} & a_{33} \\
\end{bmatrix} \quad
\begin{Bmatrix}
  a_{11} & a_{22} & a_{33} \\
  a_{21} & a_{22} & a_{23} \\
  a_{31} & a_{32} & a_{33} \\
\end{Bmatrix}\]

\section{脚注测试}

测试 \footnote{眼看他起朱楼,眼看他宴宾客,眼看他楼塌了。这青苔碧瓦堆,俺曾睡风流觉,将五十年兴亡看饱。
金粉未消亡,闻得六朝香,满天涯烟草断人肠。怕催花信紧,风风雨雨,误了春光。}

测试 \footnote[3]{君不见,左纳言,右纳史,朝承恩,暮赐死。行路难,不在水,不在山,只在人情反覆间!}


\section{引用测试}

\subsection{参考文献}

测试%\cite{whu-bachelor:1,whu-bachelor:2,whu-bachelor:3,whu-bachelor:5,whu-bachelor:7}

测试%\cite*{whu-bachelor:1,whu-bachelor:2,whu-bachelor:3,whu-bachelor:7}



\section{图表测试}

\begin{figure}[ht]
  \centering
  \includegraphics[width = 5cm]{whu-logo.pdf}
  \caption{武汉大学校徽}
  \label{fig:武汉大学校徽}
\end{figure}

引用图~\ref{fig:武汉大学校徽}

\begin{table}[ht]
  \centering
  \caption{%
    简单的表格和引用 abc 123 %\cite{whu-bachelor:1}
  }
  \label{table:简单的表格}
  \begin{tabular}{cc}
    \hline
    a & b \\ \hline
    c & d \\ \hline
    测试 & 文本 \\ \hline
  \end{tabular}
\end{table}

引用表~\ref{table:简单的表格}


\section{已定义好的一些数学定理环境}


\begin{definition}[测度]
  (参见文献xxx) 这是一段文字 $E = m c^2$  (中文括号)和 (西文括号)
\end{definition}

\begin{theorem}
  这是一段文字 $E = m c^2$
\end{theorem}


\begin{proof}
  这是一段文字 $E = m c^2$
\end{proof}

\begin{proof}[定理xx的证明]
  这是一段文字 $E = m c^2$
\end{proof}

\begin{example}
  这是一段文字 $E = m c^2$
\end{example}

\begin{property}
  这是一段文字 $E = m c^2$
\end{property}

\begin{proposition}
  这是一段文字 $E = m c^2$
\end{proposition}

\begin{corollary}
  这是一段文字 $E = m c^2$
\end{corollary}

\begin{lemma}
  这是一段文字 $E = m c^2$
\end{lemma}

\begin{axiom}
  这是一段文字 $E = m c^2$
\end{axiom}

\begin{counterexample}
  这是一段文字 $E = m c^2$
\end{counterexample}

\begin{conjecture}
  这是一段文字 $E = m c^2$
\end{conjecture}

\begin{question}
  这是一段文字 $E = m c^2$
\end{question}

\begin{claim}
  这是一段文字 $E = m c^2$
\end{claim}

\begin{remark}
  这是一段文字 $E = m c^2$
\end{remark}

\begin{theorem}[Banach-Steinhaus]\label{thm:test}
  设 $E$ 是 Banach 空间, $F$ 是赋范空间, $(u_i)_{i\in I}$ 是一族从 $E$ 到 $F$ 的有界线性算子,
  即 $(u_i)_{i\in I}\subset \mathcal{B}(E,F)$. 若对每一点 $x\in E$, 有
  $\sup_{i\in I} \|u_i(x)\|<\infty$, 则
  \begin{equation}\label{eq:test1}
    \sup_{i\in I} \|u_i\| < \infty.
  \end{equation}
\end{theorem}

我想引用定理~\ref{thm:test} 和公式~\ref{eq:test1}


定理括号测试:

\begin{theorem}
  测试
  \begin{enumerate}
    \item 中文(括号)没输入空格的效果
    \item 中文 (括号) 输入空格的效果
    \item 西文(括号)没输入空格的效果
    \item 西文 (括号) 输入空格的效果
  \end{enumerate}
\end{theorem} 


\begin{proof}
  test
  \[
    \int_{0}^{1} x^2 \d x
  \]
\end{proof}

\begin{proof}
  test
  \[
    \int_{0}^{1} x^2 \d x \qedhere
  \]
\end{proof}


\section{字体测试}
字体测试:\\
{宋体} {\heiti 黑体} {\kaishu 楷书} {\fangsong 仿宋}\\
{\rmfamily 罗马字族} {\sffamily 无衬线字族} {\ttfamily 打字机}\\
{\bfseries 粗体} {\itshape 意大利} {\slshape 倾斜}

伪粗体测试:\\
{\bfseries\songti 伪粗体} {\bfseries\kaishu 伪粗体} {\bfseries\heiti 伪粗体} {\bfseries\fangsong 伪粗体} {\bfseries 伪粗体}

伪斜体测试:\\
{\itshape\songti 伪斜体} {\itshape\kaishu 伪斜体} {\itshape\heiti 伪斜体} {\itshape\fangsong 伪斜体} {\itshape 伪斜体}

叠加测试:\\
{\bfseries\itshape 伪粗斜体} {\bfseries\sffamily 伪粗黑体} {\bfseries\ttfamily 伪粗仿宋} {\itshape\sffamily 伪斜黑体} {\itshape\ttfamily 伪斜仿宋}


% 参考文献
% \nocite{*}
% \bibliography{bachelor-refs}


\appendix

\chapter{测试}

\section{公式测试}
\section{公式测试}

\begin{equation}
  a^2 + b^2 = c^2
\end{equation}

\chapter{测试}

\section{公式测试}
\subsection{编号测试}
\subsection{编号测试}
\subsubsection{编号测试}
\subsubsection{编号测试}

\section{公式测试}

\begin{equation}\label{equation:appendix-test}
  a^2 + b^2 = c^2
\end{equation}

公式~\eqref{equation:appendix-test}

\end{document}