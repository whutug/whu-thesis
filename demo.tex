% \documentclass[type = bachelor]{whu-thesis}
\documentclass[type = master,class = academic]{whu-thesis}
% \documentclass[type = doctor]{whu-thesis}
% type: 学位类型,可选项为 bachelor, master, doctor
% class: 学位类别,可选项为 academic, professional
% showframe: 显示页面布局框架

% 以下仅列举了部分可能用到的设置选项,更多用法请参考文档《whu-thesis.pdf》

% \PassOptionsToPackage{gbnamefmt = lowercase}{biblatex} % 英文作者姓名不强制大写

\whusetup{
  info = {
    title      = {论文题目,楷体一号}, % 标题,可使用 \\ 手动换行
    title*     = {A QoS-aware Self-Adaptive System Approach in Service Computing Environment},
    department  = {数学与统计学院},
    department* = {School of Mathematics and Statistics},
    author     = {张三},
    author*    = {Xxxxxx Xxxxx},
    student-id = {2021202012345},
    supervisor  = {李某某},
    supervisor* = {Xxx Xxxxx},
    academic-title  = {教授},
    academic-title* = {Prof},
    % supervisor-outer = {王某某}, % 校外导师(非必填)
    % academic-title-outer = {高级工程师}, % 校外导师职称(非必填)
    subject = {英语}, % 学科名称(非必填)
    major   = {英国语言文学},
    major*  = {English language and literature},
    research-area  = {英语语言学},
    research-area* = {Operator Theory},
    year = 2024,
    month = 4,
    % clc = , % 分类号
    % udc = ,
    keywords  = {\LaTeX{}, 毕业论文, 模版, 武汉大学}, % 中文关键词
    keywords* = {\LaTeX{}, Thesis, Template, Wuhan~University}, % 英文关键词
  },
  style = {
    % 字体相关选项
    font = termes, % 西文字体,可选项为 default, times, xits, termes
    math-font = termes, % 数学字体,可选项为 default, xits, termes
    cjk-font = mac, % 中文字体,可选项为 windows, mac, fandol(Linux/Overleaf/TexPage), sourcehan, none
    % cjk-fakefont = true, % 使用伪粗体与伪斜体
    % 参考文献及引用相关选项
    bib-backend = bibtex, % 参考文献引擎,可选项为 bibtex, biblatex
    % bib-style = numerical, % 参考文献样式,可选项为 numerical, author-year
    % cite-style = <>, % 引用样式(自定义)
    bib-resource = {ref/bachelor-refs.bib}, % 参考文献数据源
    % 页面相关选项
    % chapter-page-header = true, % 章节首页是否有页眉
    % bachelor-encover = true, % 本科毕业论文英文封面
    % library, % 图书馆模式(去掉论文中所有的空白页)
    % license, % 使用授权协议书
    % fullwidth-stop = true, % 句号样式
    % footnote-style = <>, % 脚注编号样式
    % abstract-keywords-type  = blankline, % 摘要与关键词之间样式,可选项为 blankline, newline, vfill
    % abstract-keywords-type* = blankline, % 摘要与关键词之间样式,可选项为 blankline, newline, vfill
  }
}
\whumodule{algorithm2e}
\begin{document}
% \raggedbottom % 使得空白都置于每一页底部,可参考 https://github.com/whutug/whu-thesis/issues/276

\tableofcontents % 目录
% \listoffigures % 图目录
% \listoftables % 表目录

% 符号表
% \begin{notation}
%   $\omega_n$ & $n$-维欧氏空间中单位球的表面积 \\
%   $\alpha_n$ & $n$-维欧氏空间中单位球的体积 \\
% \end{notation}

\mainmatter

\include{pages/chapter1.tex}
% 当然你也可以直接在这里写,不过这样不太方便管理
\chapter{BBBB}


% 参考文献
% \nocite{*}
\printbibliography

% 发表的与学位论文相关的科研成果目录
% \newenvironment{achievements}{\enumerate[label={[\arabic*]}, leftmargin=*]}{\endenumerate}

\chapter*{攻博期间发表的与学位论文相关的科研成果目录}
\addcontentsline{toc}{chapter}{攻博期间发表的与学位论文相关的科研成果目录}

\section*{参与的科研项目:}

\begin{achievements}
\item 科研项目1科研项目1科研项目1科研项目1科研项目1
\item 科研项目2科研项目2科研项目2科研项目2科研项目2
\end{achievements}

\section*{发表的学术论文:}

\begin{achievements}
\item 学术论文1学术论文1学术论文1学术论文1学术论文1
\item 学术论文2学术论文2学术论文2学术论文2学术论文2
\end{achievements}

\section*{专利与软件著作权:}

\begin{achievements}
\item 专利与软件著作权1专利与软件著作权1专利与软件著作权1专利与软件著作权1专利与软件著作权1
\item 专利与软件著作权2专利与软件著作权2专利与软件著作权2专利与软件著作权2专利与软件著作权2
\end{achievements}


% 致谢
\begin{acknowledgements}
  致谢是以简短的文字对课题研究与论文撰写过程中直接给予帮助的人员(例如指导教师、答疑教师及其他人员)表示谢意。致谢是对他人劳动的尊重,也是学术规范。内容限一页。
\end{acknowledgements}

% 附录
\appendix

\include{pages/appendix.tex}

\end{document}
