% !Mode:: "TeX:UTF-8"

\documentclass[]{WHUBachelor}

\begin{document}

    \StudentNumber{2015301200xxx} %学号
    \miji{} % 密级,空着
    \Cmajor{XXX XXX} % 学科专业中文名
    \Ctitle{Altera DDR IPCore 在海量图像无级缩放硬件 实现系统中的应用} % 中文题目,请注意适当位置换行或者自动换行
    \Cauthor{XXX} % 作者
    \Csupervisor{XXX\quad 副教授} % 指导教师中文名
    \Cschoolname{XXXX学院} % 学院中文名
    \Cdate{二〇一九年五月} % 日期

    \CNabstract{
        目前对于CCD相机捕获的卫星图像的浏览和动态缩放这个比较棘手的问题的解决方案大多是通过对原始图像进行分割,然后分块显示。这些方法实现起来相对比较容易,开发成本也比较低,但是局限性非常之大,使浏览极为不便,移植性也较差。在本项目中为了解决海量图像方面的这个技术瓶颈,提出了大容量缓存加无级缩放算法的方案。
    }
    \CNkeywords{关键词1;关键词2;关键词3}

    \ENabstract{
        This paper is carried out on the basis of the 211 project-Ssmi-physical simulation system for ship motion control.
    }
    \ENkeywords{motion control; autopilot; neural; GIS}

    % !Mode:: "TeX:UTF-8"

%%% ---- 封面 ----- %%%
\setcounter{page}{1}
\begin{titlepage}
  \begin{center}
    \vspace*{10pt}
    {\heiti \zihao{5}%
      \hfill
      \newlength{\myLen}\settowidth{\myLen}{学号\ :\hspace{3mm}\the\StudentNumber\hspace{3mm}}
      \begin{minipage}[t]{\myLen}
        学号\ :\uline{\hfill\hspace{3mm}\the\StudentNumber\hspace{3mm}\hfill}  \\[2mm]
        密级\ :\uline{\hfill\the\miji\hfill}
      \end{minipage}
    }

    \par
    \vspace*{70pt} % 插入空白
    \begin{center}
      \songti \zihao{1} 武汉大学本科毕业论文 \@author
    \end{center}

    \vspace*{54pt}

    % \fbox{
    \begin{minipage}[c][6cm]{12cm}
      \centering
      \heiti \zihao{2} \the\Ctitle
    \end{minipage}

    \par
    \vspace{80pt}
    {\songti\zihao{-3}
      \newcommand\maketabox[1]{\makebox[3.5cm][s]{#1}}
      \begin{tabular}{cp{5cm}}
        \maketabox{院(系)名\ 称\ :}   & \the\Cschoolname \\[1ex]
        \maketabox{专\ 业\ 名\ 称\ :} & \the\Cmajor      \\[1ex]
        \maketabox{学\ 生\ 姓\ 名\ :} & \the\Cauthor     \\[1ex]
        \maketabox{指\ 导\ 教\ 师\ :} & \the\Csupervisor \\[1ex]
      \end{tabular}
    }
    \par
    \vspace{40pt}
    {
      {\songti \zihao{-2} \the\Cdate}
    }
  \end{center}%
  \clearpage
\end{titlepage}


%%% ---- 郑重申明页 ----- %%%
\newpage
\thispagestyle{empty}
\renewcommand{\baselinestretch}{1.5}  %下文的行距

\vspace*{44pt}
\begin{center}{\ziju{0.5}\songti \zihao{2} \textbf{郑重声明}}\end{center}
\par\vspace*{20pt}
\setlength{\baselineskip}{23pt}
{\zihao{4}
  本人呈交的学位论文,是在导师的指导下,独立进行研究工作所取得的成果,所有数据、图片资料真实可靠。尽我所知,除文中已经注明引用的内容外,本学位论文的研究成果不包含他人享有著作权的内容。对本论文所涉及的研究工作做出贡献的其他个人和集体,均已在文中以明确的方式标明。本学位论文的知识产权归属于培养单位。
  \par
  \vspace*{88pt}
  \hspace*{0.5cm}本人签名: \underline{\hspace{3.5cm}}
  \hspace{2cm}日期: \underline{\hspace{3.5cm}}\hfill\par}


\newcommand\cnkeywords[1]{ {\noindent\heiti\zihao{-4} 关键词: }\zihao{-4}#1}
\newcommand\enkeywords[1]{ {\noindent\bfseries\zihao{-4} Key words: }\zihao{-4}#1}


%%% ---- 中文摘要及关键词 ----- %%%
\newpage
\thispagestyle{empty}
\vspace{10pt}
\begin{center}{\ziju{2}\heiti \zihao{-2} 摘要}\end{center}
\baselineskip=23pt
{\songti \zihao{-4}%

  \the\CNabstract
}
\par
\vspace*{2em}
\cnkeywords{\the\CNkeywords}


%%% ---- 英文摘要及关键词 ----- %%%
\newpage
\thispagestyle{empty}
\vspace{10pt}
\begin{center}{\zihao{-2} \textbf{ABSTRACT}}\end{center}
\baselineskip=23pt
{\zihao{-4}%

\the\ENabstract
}
\par
\vspace*{2em}
\enkeywords{\the\ENkeywords}


    % 正文开始
    \mainmatter

    \chapter{绪论}
    \section{概述}
    IP(Intellectual Property)就是常说的知识产权,IPCore(知识产权核)则是指用于产品应用的专用集成电路(ASIC)或者可编程逻辑器件(PGA)的逻辑块或数据块。
    \subsection{DDR IP Core的时序性描述}
    \subsubsection{对DDR SDRAM的初始化时序}
    通过DDR IPCore 对DDR 和DDR2 SDRAM进行初始化是有分别的,由于在本次项目设计过程中实际采用的是DDR SDRAM,因此本文仅仅对前者的初始化时序进行讨论。
    
    \chapter{测试}
    \section{第三小节}
	这是一大段文字这是一大段文字这是一大段文字这是一大段文字这是一大段文字这是一大段文字这是一大段文字这是一大段文字这是一大段文字这是一大段文字这是一大段文字这是一大段文字这是一大段文字这是一大段文字这是一大段文字这是一大段文字这是一大段文字这是一大段文字这是一大段文字这是一大段文字这是一大段文字这是一大段文字这是一大段文字这是一大段文字这是一大段文字这是一大段文字这是一大段文字这是一大段文字这是一大段文字这是一大段文字这是一大段文字这是一大段文字这是一大段文字这是一大段文字这是一大段文字这是一大段文字这是一大段文字这是一大段文字这是一大段文字这是一大段文字这是一大段文字这是一大段文字这是一大段文字这是一大段文字这是一大段文字这是一大段文字这是一大段文字这是一大段文字这是一大段文字这是一大段文字这是一大段文字这是一大段文字这是一大段文字这是一大段文字这是一大段文字这是一大段文字这是一大段文字这是一大段文字这是一大段文字这是一大段文字这是一大段文字这是一大段文字这是一大段文字这是一大段文字这是一大段文字这是一大段文字这是一大段文字这是一大段文字这是一大段文字这是一大段文字这是一大段文字这是一大段文字这是一大段文字这是一大段文字这是一大段文字这是一大段文字这是一大段文字这是一大段文字这是一大段文字这是一大段文字这是一大段文字这是一大段文字这是一大段文字这是一大段文字这是一大段文字这是一大段文字这是一大段文字这是一大段文字这是一大段文字这是一大段文字这是一大段文字这是一大段文字这是一大段文字这是一大段文字这是一大段文字这是一大段文字这是一大段文字这是一大段文字这是一大段文字这是一大段文字这是一大段文字这是一大段文字这是一大段文字这是一大段文字这是一大段文字这是一大段文字这是一大段文字这是一大段文字这是一大段文字这是一大段文字这是一大段文字这是一大段文字这是一大段文字这是一大段文字这是一大段文字这是一大段文字这是一大段文字这是一大段文字这是一大段文字这是一大段文字这是一大段文字这是一大段文字这是一大段文字这是一大段文字这是一大段文字这是一大段文字这是一大段文字这是一大段文字这是一大段文字这是一大段文字这是一大段文字这是一大段文字这是一大段文字这是一大段文字这是一大段文字这是一大段文字这是一大段文字这是一大段文字这是一大段文字
	\section{新的大节}
	新的大节会自动出现在新的一页上
	\section{参考文献和交叉引用}\label{sec:ref}
	\subsection{参考文献}
    这是一个参考文献引用的范例\cite{Stone_1998}

    % 生成参考文献
    % 使用方法:\bibliography{参考文件1文件名, 参考文献2文件名, ...}
    \bibliography{ref/refs}

\end{document}