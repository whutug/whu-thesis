% Chapter 1

\chapter{绪论}
\section{模板概述}
\WhuThesis{}(\textbf{W}u\textbf{h}an \textbf{U}niversity \LaTeX{} \textbf{Thesis} Template)是为了帮助武汉大学毕业生撰写毕业论文(设计)而编写的 \LaTeX{} 模板,现时段暂时只提供本科生毕业论文(设计)模板。

模板根据《\href{https://github.com/mtobeiyf/whu-thesis/files/4638713/default.pdf}{武汉大学本科生毕业论文(设计)书写印制规范}》编写,力求合规,简洁,易于实现,用户友好。

与 \app{Word} 等所见即所得编辑工具不同,使用 \LaTeX{} 工具排版可以将写作与排版过程分离,写作者只需要关心文字的部分,而剩下的排版工作全部交给工具自动完成。

本手册假定用户已经能处理一般的 \LaTeX{} 文档,并对 \hologo{BibTeX} 有一定了解。如果从未接触过 \TeX{} 和 \LaTeX{},建议先学习相关的基础知识。

\begin{notice}
    模板的作用在于减少论文写作过程中格式调整的时间。前提是遵守模板的用法,否则即便用了 \WhuThesis 也难以保证输出的论文符合学校规范。
\end{notice}

用户如果遇到 bug,或者发现与学校《印制规范》的要求不一致,可以尝试以下办法:
\begin{enumerate}
    \item 阅读学校的\href{https://github.com/mtobeiyf/whu-thesis/files/4638713/default.pdf}{书写印制规范文件},判断是否符合要求;
    \item 前往项目 \href{https://github.com/mtobeiyf/whu-thesis/wiki}{wiki} 查看相关说明;
    \item 将 \TeX{} 发行版和宏包升级到最新,并且将模板升级到 Github 上最新版本,查看问题是否已经修复;
    \item 在 \href{https://github.com/mtobeiyf/whu-thesis/issues}{GitHub Issues 页面}中搜索该问题的关键词;
    \item 提出新的 \href{https://github.com/mtobeiyf/whu-thesis/issues}{issue},并说明系统、\TeX{} 版本、出现的问题等关键信息。
\end{enumerate}

\section{模板选项}
模板共提供了 \verb|degree| 与 \verb|class| 两类选项,其中 \verb|degree| 下有 \verb|bachelor|(默认)、\verb|master|、\verb|doctor| 三选项,\verb|class| 下有 \verb|paper|(默认)、\verb|design|、\verb|manual| 三选项,合计两类六选项。

\verb|degree| 选项用于学位选择。虽然提供了 \verb|bachelor|、\verb|master|、\verb|doctor| 三个选项,但现时段只有 \verb|bachelor| 可以使用,也即只提供本科毕业论文(设计)模板。

\verb|class| 选项用于文档类型。其中 \verb|paper| 与 \verb|design| 的区别只在于封面显示的是“武汉大学毕业论文”还是“武汉大学毕业设计”。此功能只是实验性功能,《印制规范》中并无相关要求。\verb|manual| 选项则只用于本手册编写,不用于论文的实际撰写过程。

在使用模板选项时,既可以使用 \meta{key} = \meta{value} 格式,也可以直接输入选项名。

另外,虽然标准文档类的选项亦可以使用,但只推荐视指导老师要求使用 \verb|oneside| 与 \verb|twoside|(默认)两选项之一。\verb|twoside| 选项启用时,各章会在奇数页(右边)开始。

\section{格式要求}
正文字号宋体小四,正文行间距固定为 23 点(point,pt,\app{Word} 中译作“磅”)。

空格键和 Tab       键输入的空白字符视为“空格”。连续的若干个空白字符视为一个空。一行开头的空格忽略不计。\par
行末的回车视为一个空格;但连续两个回车,也就是空行,会将文字分段。多个空行被视为一个空行。也可以在行末使用 \verb|\par| 命令分段。



使用 \verb|%| 进行注释。在这个字符之后直到行末,所有的字符都被忽略,行末的回车也不引入空格。% 我是注释

\section{各节一级标题}
我是内容

\subsection{各节二级标题}
你是内容

\subsubsection{各节三级标题}
他是内容

\section{字体字号}
与 \CTeX 文档类不同,\WhuThesis 只定义了中文的宋体与{\heiti 黑体},而并未定义楷体与仿宋体。出于多方面考量,宋体与黑体也只使用了 Windows 平台下的中易宋体与中易黑体,Linux 用户请将中易宋体与中易黑体文件(simsun.ttc 与 simhei.ttf)放置在工作目录下。作为对 \app{Word} 的模仿,文档内亦可对宋体使用\textbf{伪粗体}与\textit{伪斜体},在此之上,两者可组合形成\textbf{\textit{粗斜体}}。

\begin{notice}
    《印制规范》内并未对西文的无衬线(\textsf{sans serif})字体与等宽(\texttt{mono})字体进行要求,但 \WhuThesis 并未将其显式地将其全部定义为 Times New Roman。手册中的无衬线体与等宽字体仍使用 \LaTeX{} 的默认无衬线与等宽字体。在下一段建议的基础上,用户可以使用 \verb|\setsansfont| 与 \verb|\setmonofont| 命令对无衬线体与等宽字体进行更改与使用。

    除非你非常清楚自己在干什么,否则不要轻易改变字体。当然\\
    \begin{quote}
        文档\textbf{内}{\tiny 使用}的字体越{\huge \textsf{多}},\textit{文档}就越具有\textbf{\textit{可读性}}{\large 与}{\Large 美}{\huge 观}{\Huge 性}。
    \end{quote}
\end{notice}

\section{编译}
本模板必须使用 \hologo{XeLaTeX} 或 \hologo{LuaLaTeX} 配合 \hologo{BibTeX} 编译,否则会直接报错。一般来说,\hologo{XeLaTeX} 引擎的编译速度较快且占用资源较少,而 \hologo{LuaLaTeX} 引擎的编译结果似乎有更好的跨平台规范性。配合 \pkg{lua-visual-debug} 包,可以得到更详细的编译结果。若使用 LuaHB\TeX{} 引擎编译,配合 \pkg{emoji} 包,还可进一步使用 Emoji 等功能。\emojitest 本模板支持多个平台,结合 \app{Sublime}、\app{VS Code} 或 \app{Overleaf} 等都可以使用。编译之前请确保已经给出了学号 \verb|\stunum|、题目 \verb|\title|、院系 \verb|\school|、专业 \verb|\major|、作者 \verb|\author|、指导教师 \verb|\advisor| 及职称 \verb|\proftitle| 等七个必需信息,以及日期 \verb|\date| 一个可选信息。必需信息缺失时,模板会报错,日期信息缺失时,默认为汉数字形式的编译时间点的年月,即\zhdigits{\the\year}年\zhnumber{\the\month}月。

\begin{notice}
    由于 \CTeX 宏集会根据用户使用的编译方式,在底层选择不同的中文支持方式。因此不同的编译方式和中文支持方式会在一定程度上影响 \CTeX 宏集的行为,如对空格、标点的处理等。一般来说,使用 \hologo{XeLaTeX} 编译时,推荐在中西文间显式地插入一个西文空格,而使用 \hologo{LuaLaTeX} 编译时中西文间不插入空格。
\end{notice}

\subsection{latexmk}
\verb|latexmk| 命令支持全自动生成 \LaTeX{} 编写的文档,并且支持使用不同的工具链来进行生成,它会自动运行多次工具直到交叉引用都被解决。编译链如下所示。

如果使用 \hologo{XeLaTeX} 进行编译,编译命令为
\lstset{basicstyle=\ttfamily, breaklines=true}
\begin{lstlisting}[language=bash]
$ latexmk -xelatex main.tex
\end{lstlisting}

如果使用 \hologo{LuaLaTeX} 进行编译,编译命令为
\lstset{basicstyle=\ttfamily, breaklines=true}
\begin{lstlisting}[language=bash]
$ latexmk -lualatex main.tex
\end{lstlisting}

\subsection{\hologo{XeLaTeX}(或 \hologo{LuaLaTeX}) + \hologo{BibTeX}}
用户也可以直接使用 \hologo{XeLaTeX} 或 \hologo{LuaLaTeX} 配合 \hologo{BibTeX} 进行编译,编译链如下所示(可以省略扩展名)。
\begin{lstlisting}[language=bash]
$ xelatex main.tex  
$ bibtex main.aux
$ xelatex main.tex
$ xelatex main.tex
\end{lstlisting}
或
\begin{lstlisting}[language=bash]
$ lualatex main.tex  
$ bibtex main.aux
$ lualatex main.tex
$ lualatex main.tex
\end{lstlisting}

在特殊情况下,可能需要在编译时加入 \verb|-shell-escape| 选项,如
\begin{lstlisting}[language=bash]
$ xelatex -shell-escape main.tex 
\end{lstlisting}

如果用户使用 \app{vscode} 进行写作,在目录内的\ .vscode 文件夹内已经有配置好的 latexmk 与 \hologo{XeLaTeX}、\hologo{LuaLaTeX} 编译链,用户可直接使用。