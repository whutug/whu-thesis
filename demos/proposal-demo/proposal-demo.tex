\documentclass[type = proposal]{../../whu-thesis}

\whusetup{
  info = {
    title      = {武汉大学本科开题报告与任务书 \LaTeX{} 模板},
    department = {数学与统计学院},
    student-id = {20212020123456},
    author     = {夏大鱼羊}
  }
}

\usepackage{zhlipsum}  % 测试假文


\begin{document}

%%% 如果「任务书」和「开题报告」都需要编译参考文献的话
%%    不要同时编译,而是把下面的 \input 注释其中一个之后
%%    用 bibtex 编译链进行编译。然后转存生成的 PDF(不然会被覆盖)
%%    然后注释掉另一个,重新用 bibtex 编译链编译。

%%% 「任务书」和「开题报告」的数据库分别为 refs-tasks.bib 和 refs-report.bib



% 武汉大学本科毕业论文(设计)任务书
% % !TeX root = ../proposal-demo.tex

% 武汉大学本科毕业论文(设计)任务书
\ProposalTasks
% 个人信息
\ProposalTasksInfomation


\section{毕业论文(设计)题目的来源}

\zhlipsum[1]


\section{毕业论文(设计)应完成的主要内容}

\zhlipsum[1]


\section{毕业论文(设计)的基本要求及应完成的成果形式}

\zhlipsum[1]


\section{毕业论文(设计)的进度安排}

\zhlipsum[1]


\section{毕业论文(设计)应收集的资料及主要参考文献}

测试 \cite{whu-bachelor:1}

\nocite{*}

\bibliography{refs-tasks}


\section{其他要求}
\zhlipsum[1-2]



% 落款
\ProposalTasksSignature

% 开题报告
% !TeX root = ../proposal-demo.tex

% 开题报告
\ProposalReport
% 个人信息
\ProposalReportInfomation


\section{论文选题的目的和意义}

\zhlipsum[1-2]


\section{国内外关于该论题的研究现状和发展趋势}

\zhlipsum[1]


\section{论文的主攻方向、主要内容、研究方法及技术路线}

\zhlipsum[1]


\section{论文工作进度安排}

\zhlipsum[1]


\section{论文主要文献}

测试 \cite{4.1.2:1}

\nocite{*}

\bibliography{refs-report}



% 落款
\ProposalReportSignature


\end{document}