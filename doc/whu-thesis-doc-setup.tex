\usepackage{xeCJKfntef, xpinyin}
\usepackage{graphicx}
\usepackage{tabularray}
\usepackage{circledtext}
\usepackage{exam-zh-choices}

\graphicspath{{figures}}

\hypersetup{
  pdftitle  = {whu-thesis},
  pdfauthor = {夏康玮}
}
% 全角标点放在引号中,需要改成半角式,否则间距过大,不好看
\def\FSID{“{\xeCJKsetup{PunctStyle=banjiao}。}”} % U+3002
\def\FSFW{“{\xeCJKsetup{PunctStyle=banjiao}.}”} % U+FF0E
\def\COFW{“{\xeCJKsetup{PunctStyle=banjiao}:}”} % U+FF1A
\def\SCFW{“{\xeCJKsetup{PunctStyle=banjiao};}”} % U+FF1B


\title{\textcolor{MaterialIndigo800}{%
  \textbf{whu-thesis: 武汉大学学位论文 \LaTeX \xpinyin[font=\sffamily,format=\color{MaterialIndigo800}]{模}{mu2}板}}}
\author{whutug%
  \thanks{%
    \url{https://github.com/whutug}
  }
}
\date{2022/08/04\quad v0.0.1%
  \thanks{%
  \url{https://github.com/whutug/whu-thesis}
  }
}


\ExplSyntaxOn
\RenewDocumentCommand \emph { m }
  {
    \group_begin:
      \bfseries 
      \CJKunderline*{#1}
    \group_end:
  }

\NewDocumentEnvironment { reference } { O{\ttfamily} +b }
  {
    \par
    #1
    #2
  }{\par}

\NewDocumentEnvironment { points } { O{要点} +b }
  {
    \par
    \addvspace { .5em plus .5em }
    \noindent \textbf{#1}:
    \begin { enumerate }
      [
        leftmargin = 0pt ,
        itemindent = 3.5em ,
        labelsep   = 0.5em ,
        labelwidth = 1.5em ,
        align      = right ,
        label      = { \arabic * . } ,
      ]
      #2
  }
  {
    \end { enumerate }
    \par \addvspace { 1em plus .5em }
  }
\ExplSyntaxOff

\setlist[itemize, 1]
  {
    leftmargin = 3em
  }