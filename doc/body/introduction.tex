% !TeX root = ../whu-thesis-doc.tex

\section{介绍}

目前在网上可搜索到的武汉大学学位论文 \LaTeX{} 模版主要有:

\begin{enumerate}
  \item 武汉大学 \href{http://aff.whu.edu.cn/huangzh/}{黄正华老师} 编写的 \href{http://aff.whu.edu.cn/huangzh/#:~:text=%E4%B8%8B%E5%88%97-,%E6%AF%95%E4%B8%9A%E8%AE%BA%E6%96%87%E6%A8%A1%E6%9D%BF,-%2C%20%E9%80%82%E7%94%A8%E4%BA%8E%20TeX}{模版} (最近更新为 2016 年);
  \item \href{https://github.com/imfing}{imfing} 于 2019 年首先开发了 \cls{whu-thesis},\href{https://github.com/T0nyX1ang}{Tony Xiang} 参与小部分开发,最后 \href{https://github.com/tanukihee}{ListLee} 于2020 年 \href{https://github.com/whutug/whu-thesis/commit/d488438b7819ddf5a128081a50b118d8fd4ec1ef}{用 \LaTeX3 重构了 \cls{whu-thesis}},但由于维护者工作原因,于 2021 年六月份后无人继续维护 \cls{whu-thesis};
  \item \href{https://github.com/BenjaminHb/whu-thesis}{BenjaminHb} 针对学术硕士需求,基于 whu-tug 编写的 \href{https://github.com/whutug/whu-thesis}{\cls{whu-thesis}} (本模版2022年重构前的 v0.6d 版本)进行调整。
\end{enumerate}

但是黄老师的模版开发时间较早,部分配置可能已经过时,且并没有统一本硕博三个模版;BenjaminHb 对学术硕士的需求单独调整完全可以通过代码的调整在 \cls{whu-thesis} 中实现(比如提供键值接口);而 \cls{whu-thesis} 在 2021 年 6 月后无人维护。

随着关于 \cls{whu-thesis} 的 \href{https://github.com/whutug/whu-thesis/issues}{问题} 和 \href{https://github.com/whutug/whu-thesis/discussions}{讨论} 不断增多,模版的优化需求日渐强烈。\href{https://github.com/xkwxdyy}{xdyyxkw} 在 2022 年接手 \cls{whu-thesis} 的维护后,于 2022 年 8 月重新梳理教务处规范、已有的部分学院的需求差异,基于黄正华老师的模板、v0.6d 版本的 \cls{whu-thesis} 与 issues 和 discussions 中的问题和需求,对 \cls{whu-thesis} 进行整体重构工作。(从此之后版本号风格变成 v$x.y.z$)

本模板将借鉴前辈经验,重新设计,并使用 \LaTeX3 编写,以适应 \TeX{} 技术发展潮流;同时还将构建一套简洁的接口,方便用户使用。主要参考:复旦大学 \cls{fduthesis}、清华大学 \cls{thuthesis}、中国科学技术大学 \cls{ustcthesis}、中国科学院大学 \cls{ucasthesis} 以及上海交通大学 \cls{sjtuthesis} 等成熟的学位论文 \LaTeX{} 模版,同时也会参考 xkwxdyy 于 2022 年开发的华中师范大学学位论文模版 \cls{CCNUthesis},并吸取模版开发过程中的教训,总结经验,努力开发一个优质的 \cls{whu-thesis}。
